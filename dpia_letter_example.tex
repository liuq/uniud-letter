\documentclass[dpia,subject=true]{scrlttr2}
%\usepackage[italian]{babel}
\usepackage{polyglossia}
\setmainlanguage{italian}
\usepackage{lipsum}
\usepackage{metalogo}
\usepackage{listings}

\definecolor{Brown}{cmyk}{0,0.81,1,0.60}
\definecolor{OliveGreen}{cmyk}{0.64,0,0.95,0.40}
\definecolor{CadetBlue}{cmyk}{0.62,0.57,0.23,0}
\lstset{
basicstyle=\ttfamily\small,
keywordstyle=\color{OliveGreen},
identifierstyle=\color{CadetBlue}\bfseries,
stringstyle=\color{blue},
commentstyle=\color{Brown},
morecomment=[l][\color{Brown}]{\#},
breaklines=true,
language=TeX
}

\setkomavar{fromname}{Prof. Luca Di Gaspero}
\setkomavar{signature}{Luca Di Gaspero}
\setkomavar{fromphone}{+39 0432 55 8242}
\setkomavar{frommobilephone}{+39 320 4366035}
\setkomavar{fromemail}{luca.digaspero@uniud.it}
\setkomavar{fromurl}{www.dpia.uniud.it/digaspero}

\begin{document}
  \begin{letter}{Spett.li Signori Utenti \LaTeX \\
    dell'Università degli Studi di Udine \\
    \MakeUppercase{Loro Sedi}
    }
    \setkomavar{subject}{nuovo stile \LaTeX Uniud corporate image}
    \opening{Gentili signore e egregi signori,}

è con gran piacere che vi presento il nuovo stile \LaTeX{} per la corrispondenza istituzionale basato sul manuale d'immagine della nostra Università.

    Lo stile è modulare e personalizzabile ed è basato sulla classe \texttt{scrlttr2} del pacchetto \texttt{KOMAscript}. Per motivi legati all'utilizzo dei font (in particolare il package \texttt{fontspec}) è necessario utilizzare \XeLaTeX{} per la compilazione.
    
    Per scrivere una lettera usando lo stile è sufficiente impostare alcune variabili autoesplicative.
\begin{lstlisting}
\setkomavar{fromname}{Prof. Luca Di Gaspero}
\setkomavar{fromphone}{+39 0432 55 8242}
\setkomavar{frommobilephone}{+39 320 4366035}
\setkomavar{fromemail}{luca.digaspero@uniud.it}
\setkomavar{fromurl}{www.dpia.uniud.it/digaspero}
\end{lstlisting}

    Volendo modificare la firma (in modo da evitare il titolo indicato, invece, nell'intestazione) è possibile impostare la variabile \texttt{signature} nel modo seguente \lstinline|\setkomavar{signature}{Luca Di Gaspero}|.

    Inoltre, è possibile, utilizzando i package \texttt{babel} o \texttt{polyglossia}, impostare la lingua del documento:
\begin{lstlisting}
\usepackage[italian]{babel}  
\end{lstlisting}
    oppure
\begin{lstlisting}
\usepackage{polyglossia}
\setmainlanguage{english}
\end{lstlisting}    

    Per preimpostazione viene visualizzato, sul lato destro, il logo dell'accreditamento ISO-9001 d'Ateneo, qualora lo si volesse rimuovere è sufficiente, nel prologo, dare il comando:
\begin{lstlisting}
\setkomavar{qualitysystem}{}
\end{lstlisting}
  e il logo verrà sostituito dal più famigliare richiamo al web d'ateneo \textbf{www.uniud.it}.

    È possibile, inoltre, modificare la struttura di afferenza e gli indirizzi, impostando le seguenti variabili nel prologo o, come vedremo in seguito, in un file di stile personalizzato:
\begin{lstlisting}
\setkomavar{fromdepartment}{\IfLanguagePatterns{italian}{Dipartimento politecnico di Ingegneria e Architettura}{Polytechnic Department of Engineering and Architecture}}
\setkomavar{fromaddress}{\IfLanguagePatterns{italian}{via delle Scienze 206 \\ 33100 Udine}{via delle Scienze 206 \\ I-33100 Udine, Italy}}
\end{lstlisting}
  si osservi l'uso del comando \lstinline|IfLanguagePatterns{italian}{T}{F}| che consente di modificare il testo a seconda della lingua (il default è che venga utilizzata la dizione inglese se non è specificata esplicitamente la lingua italiana).
  
  Volendo creare una versione personalizzata della lettera per una struttura diversa, è sufficiente definire un nuovo file \texttt{nomestruttura.lco} contenente il codice simile al seguente:
\begin{lstlisting}
\ProvidesFile{dpia.lco}[2017/02/01 lco (Luca Di Gaspero)]
\LoadLetterOption{uniud}
% DPIA customization
\setkomavar{fromdepartment}{\IfLanguagePatterns{italian}{Dipartimento politecnico di Ingegneria e Architettura}{Polytechnic Department of Engineering and Architecture}}
\setkomavar{fromaddress}{\IfLanguagePatterns{italian}{via delle Scienze 206 \\ 33100 Udine}{via delle Scienze 206 \\ I-33100 Udine, Italy}}
\end{lstlisting}
   in particolare in questo caso si sono effettuate le personalizzazioni relative al Dipartimento Politecnico di Ingegneria e Architettura.
  
    Per usare lo stile è necessario installare sul proprio sistema i font \texttt{Helvetica Neue},  in particolare nella variante \textbf{heavy}, indispensabili per comporre l'intestazione e il pié di pagina secondo le specifiche del manuale di immagine.
    
    Qualora ci fossero problemi nell'utilizzo di questo modello non esitate a contattarmi.

    \closing{Cordialità,}
    \vfill
    \ps{P.S. Riporto di seguito lo schema di documento da cui è stata ottenuta questa lettera.}
\begin{lstlisting}
\documentclass[dpia,subject=true]{scrlttr2}
\usepackage[italian]{babel}

\setkomavar{fromdepartment}{\IfLanguagePatterns{italian}{Dipartimento politecnico di Ingegneria e Architettura}{Polytechnic Department of Engineering and Architecture}}
\setkomavar{fromaddress}{\IfLanguagePatterns{italian}{via delle Scienze 206 \\ 33100 Udine}{via delle Scienze 206 \\ I-33100 Udine, Italy}}

\setkomavar{fromname}{Prof. Luca Di Gaspero}
\setkomavar{fromphone}{+39 0432 55 8242}
\setkomavar{frommobilephone}{+39 320 4366035}
\setkomavar{fromemail}{luca.digaspero@uniud.it}
\setkomavar{fromurl}{www.dpia.uniud.it/digaspero}

\begin{document}
  \begin{letter}{Spett.li Signori Utenti \LaTeX \\
    dell'Università degli Studi di Udine \\
    \MakeUppercase{Loro Sedi}
    }
    \setkomavar{subject}{nuovo stile \LaTeX Uniud corporate image}
    \opening{Gentili signore e egregi signori,}
    ...
    \closing{Cordialità,}
    \ps{P.S.  Riporto di seguito lo schema di documento da cui è stata ottenuta questa lettera.}
    ...
    \vfill
    \encl{Non ce ne sono}
  \end{letter}
\end{document}
\end{lstlisting}
    \vfill
    \encl{Non ce ne sono}
  \end{letter}
\end{document}